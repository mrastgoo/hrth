\label{generalpipeline}
%    text width=4.5em, text badly centered, node distance=3cm, inner sep=0pt]
\tikzstyle{block} = [rectangle, draw, fill=azulunam, drop shadow, text = white,
    text width=7em, text centered, rounded corners, minimum height=3em]
    \tikzstyle{myarrow}=[->, thick]
    \tikzstyle{line}=[-, thick]
\tikzstyle{cloud} = [draw, ellipse,fill=azulunam!20, drop shadow,
    minimum height=2em]
\tikzstyle{block2}= [rectangle, draw, fill=azulunam!20, drop shadow, text = black, text width=5em, text centered, rounded corners, minimum height=3em]
    % [node distance =1cm, auto]
    \begin{center}
\begin{tikzpicture}[node distance = 0.7cm,scale=0.7, every node/.style={scale=0.7}]
    % Place nodes
    %\node [block] (init) {initialize model};
    \node [block2] (input) {Image};
    \node [block, right of=input, node distance = 10cm](PreProcessing){Pre-Processing};
    \node [block, right of=PreProcessing, node distance = 10cm](FtEx){Feature Extraction};
    \node [block, below of=FtEx, node distance = 5cm](DR){Feature Modification};
    \node [block, below of = PreProcessing, node distance = 5cm](clsfy){Classification};
    \node [block2, below of =input, node distance = 5cm] (output){Label};

    % Draw edges
    \draw [myarrow] (input) -- (PreProcessing);
    \draw [myarrow] (PreProcessing) -- (FtEx);
    \draw [myarrow] (FtEx) -- (DR);
    \draw [myarrow] (DR) --(clsfy); 
    \draw [myarrow] (clsfy) --(output);

\end{tikzpicture}
\end{center}

%\frame{
%\frametitle{Related Work}
%
%\begin{columns}
%    \begin{column}{0.2\textwidth}
%       
%	 \scriptsize{
%	 
%	Stated Classification Results in the literature\\
%	References {\color{green} green}\\
%	Sensitivity {\color{blue}blue}\\
%	Specificity black\\
%	 Data size \color{red}red}
%    \end{column}
%    \begin{column}{0.8\textwidth}
%		\input{Figure1.tex}
%    \end{column}
%  \end{columns}
%}
%
%%---------------
%\frame{
%\frametitle{Related Work}
%	\scriptsize{Extracted features from dermoscopic images. The highlighted references in {\color{blue}blue} color used local features (e.g., bag of features or bag of words).}
%\input{Table1.tex}
%}


